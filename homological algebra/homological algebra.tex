% ----------------------------------------------------------------
% Report Class (This is a LaTeX2e document)  *********************
% ----------------------------------------------------------------
\documentclass[11pt]{report}
\usepackage[english]{babel}
\usepackage{amsmath,amsthm}
\usepackage{amsfonts}
\usepackage{tikz-cd}
% THEOREMS -------------------------------------------------------
\newtheorem{thm}{Theorem}[chapter]
\newtheorem{cor}[thm]{Corollary}
\newtheorem{lem}[thm]{Lemma}
\newtheorem{ex}[thm]{Exercise}
\newtheorem{prop}[thm]{Proposition}
\theoremstyle{definition}
\newtheorem{defn}[thm]{Definition}
\theoremstyle{remark}
\newtheorem{rmk}[thm]{Remark}
% ----------------------------------------------------------------

%-----------------------------------------------------------------
\newcommand{\reals}{\mathbb{R}}
\newcommand{\interval}{[0,1]}
\newcommand{\gradient}{\mathrm{grad}}
\newcommand{\rotation}{\mathrm{rot}}
\newcommand{\divergence}{\mathrm{div}}
\newcommand{\kernel}{\mathrm{ker}}
\newcommand{\image}{\mathrm{im}}
\newcommand{\derive}{\mathrm{d}}

\begin{document}

\chapter{Introduction}\label{start_question}
Here, I will do the exercises in the book \emph{from calculus to  cohomology}.
\begin{ex}
Is there a smooth function $F:U\to\reals$, such that
\[
\frac{\partial F}{\partial x_1}=f_1, ~ \frac{\partial F}{\partial x_2}= f_2, ~ f=(f_1,f_2)
\]
\end{ex}
\begin{proof}
  \[
  \frac{\partial^2F}{\partial x_1\partial x_2}=\frac{\partial^2 F}{\partial x_2\partial x_1}
  \implies \frac{\partial f_1}{\partial x_2}=\frac{\partial f_2}{\partial x_1}
  \]
\end{proof}
Is the above condition sufficient?
\begin{ex}
  The answer is no.
  
  For example, we set $f=(f_1,f_2)$ as follows:
  \[
  f_1=\frac{-x_2}{x_1^2+x_2^2},~ f_2=\frac{x_1}{x_1^2+x_2^2}
  \]
\end{ex}
\begin{proof}
  \[
  \frac{\partial f_1}{\partial x_2}=\frac{\partial f_2}{\partial x_1}=\frac{x_2^2-x_1^2}{(x_1^2+x_2^2)^2}
  \]
\end{proof}
\begin{defn}
  A subset $X\subset\reals^n$ is said to be \emph{star-shaped} with respect to the point $x_0\in X$ is the line segment $\{tx_0+(1-t)x\mid t\in\interval\}$ is contained in $X$ for all $x\in X$.
\end{defn}
\begin{ex}
  If $X$ is a star-shaped space, then the solution of Exercise \ref{start_question} is affirmative.
\end{ex}
\begin{proof}
  WLOG, $x_0=0$. Let $G(t)=F(tx_1,tx_2)$, then
  \[
  G(t)=\int_0^t\frac{\partial G(s)}{\partial s}ds=\int_0^t x_1f_1(sx_1,sx_2)+x_2f_2(sx_1,sx_2) ds
  \]
  as we desired.
\end{proof}
\begin{rmk}
Star-shaped space is ``contractible" (topological property).
\end{rmk}
\begin{defn}
  Let $U\subset\reals^k$ and $C^\infty(U,\reals^k)$ be the set of smooth functions $\phi:U\to \reals^k$. For $k=2$, we define the \emph{gradient} and \emph{rotation}
  \[
  \gradient: C^\infty(U,\reals)\to C^\infty(U,\reals^k),\qquad \rotation:C^\infty(U,R^2)\to C^\infty(U,\reals)
  \]
\end{defn}
\begin{prop}
  \[\rotation\circ\gradient=0\]
\end{prop}
\begin{defn}
  \[
  H^1(U)=\kernel(\rotation)/\image(\gradient)
  \]
\end{defn}
\begin{rmk}
  We have the following fact:
  \[
  H^1\left(\reals^2-\bigcup_{i=1}^k\{x_i\}\right)\cong \reals^k
  \]
  \[
    \implies h^1\left(\reals^2-\bigcup_{i=1}^k\{x_i\}\right)=\#\{\text{holes}\}
  \] 
\end{rmk}
\begin{defn}
  We can define the \emph{gradient} for $U\subset\reals^k$ with $k\geq 1$ as follows
  \[
  \gradient(f)=\left(\frac{\partial f}{\partial x_1},\cdots,\frac{\partial f}{\partial x_n}\right)
  \]
\end{defn}
\begin{defn}
  \[
  H^0(U)=\kernel\left(\gradient\right)
  \]
\end{defn}
\begin{rmk}
  \[
  \begin{tikzcd}
    0\ar[r]&C^\infty(U,\reals)\ar[r,"\gradient"]&C^\infty(U,\reals^2)\ar[r,"\rotation"]&
    C^\infty(U,\reals)
  \end{tikzcd}
  \]
\end{rmk}
\begin{defn}
  An open set $U\subset \reals^k$ is connected if and only if $H^0(U)=\reals$.
\end{defn}
\begin{proof}
  If $\gradient(f)=0$, then $f$ is locally constant. Hence, $f^{-1}\left(f(x_0)\right)$ is open and closed, hence, the connected component of $U$, which is exactly $U$ itself.
\end{proof}
\begin{rmk}
\[
h^0(U)=\#\{\text{the component of }U \}
\]
\end{rmk}
\begin{defn}
  When $k=3$, we define the concept of \emph{gradient, rotation, divergence}.
  \begin{align*}
    \gradient: &C^\infty(U,\reals)\to C^\infty(U,\reals^3) \\
    \rotation:&C^\infty(U,\reals^3)\to C^\infty(U,\reals^3) \\
    \divergence:&  C^\infty(U,\reals^3)\to C^\infty(U,\reals)
  \end{align*}
  which is defined by
  \begin{align*}
     \gradient(f)&=\left(\frac{\partial f}{\partial x_1},\frac{\partial f}{\partial x_2},\frac{\partial f}{\partial x_3}\right) \\
     \rotation(f_1,f_2,f_3)&=\left(\frac{\partial f_3}{\partial x_2}-\frac{\partial f_2}{\partial x_3},\frac{\partial f_1}{\partial x_3}-\frac{\partial f_3}{\partial x_1},\frac{\partial f_2}{\partial x_1}-\frac{\partial f_1}{\partial x_2}\right)\\
    \divergence(f_1,f_2,f_3)&=\frac{\partial f_1}{\partial x_1}+\frac{\partial f_2}{\partial x_2}+\frac{\partial f_3}{\partial x_3}.
  \end{align*}
\end{defn}
\begin{prop}
The sequence below
  \[
  \begin{tikzcd}
        0\ar[r]&C^\infty(U,\reals)\ar[r,"\gradient"]& C^\infty(U,\reals^3)\ar[r,"\rotation"]&C^\infty(U,\reals^3)\ar[r,
        "\divergence"]&C^\infty(U,\reals)
  \end{tikzcd}
  \]
  is a complex, i.e.
  \[
  \rotation\circ\gradient=0,\qquad \divergence\circ\rotation=0
  \]
\end{prop}
\begin{defn}
  \[
  H^2(U)=\kernel(\divergence)/\image(\rotation).
  \]
\end{defn}
\begin{thm}
  For an open star-shaped set in $\reals^3$ we have that $H^0(U)=\reals$, $H^1(U)=0$ and $H^2(U)=0$.
\end{thm}
\begin{proof}
  First, $\gradient(f)=0$ implies that $f$ is locally constant.
  
  Assume $x_0=0$ again, then
  we can define a function 
  \[
  G(t)=F(tx_1,tx_2,tx_3)
  \]
  \[
  G(t)=\int_0^t \left(x_1f_1+x_2f_2+x_3f_3\right) ds
  \]
  Then $F(x_1,x_2,x_3)=G(1)$
  
\begin{align*}
\left.\frac{\partial F}{\partial x_1}\right\vert_{(x_1,x_2,x_3)}&=\int_0^1 \left(f_1+tx_1\frac{\partial f_1}{\partial x_1}+tx_2\frac{\partial f_2}{\partial x_1}+tx_3\frac{\partial f_3}{\partial x_1}\right) dt\\
&=\int_0^1 \left(f_1+tx_1\frac{\partial f_1}{\partial x_1}+tx_2\frac{\partial f_1}{\partial x_2}+tx_3\frac{\partial f_1}{\partial x_3}\right) dt\\
&=\int_0^1 \frac{\derive}{\derive t} \left(tf_1(tx_1,tx_2,tx_3)\right) dt\\
&=\left.tf_1(tx_1,tx_2,tx_3)\right\vert_{t=0}^1\\
&=f(x_1,x_2,x_3)
\end{align*}

For $H^2(U)$,
\end{proof}
\chapter{Simplicial sets}
\section{Triangulated spaces}
\subsection{Main Definition}

\end{document}
% ----------------------------------------------------------------
